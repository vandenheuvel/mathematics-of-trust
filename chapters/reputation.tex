\chapter{Reputation}\label{chapter:reputation}
\section{Intuition}
It is quite easy to describe intuitively what we want the reputation function $r$ to achieve. 
It should reward good behavior, 
such as contributing to those who contribute, 
and punish behavior that threatens the health of the system, 
such as free riding. 
The difficulty is of course how to formalize the different behaviors.

\section{Requirements and examples}\label{section:requirements_reputation}
We will first describe several properties which the reputation function should at least respect if we want to achieve finite leakage. 

\subsection{Giving feedback}
Any reputation function must be in some way recursive: it must give a high reputation to those contributing to the agents with a high reputation. If this would not be the case, a high reputation is not valuable and as such not something that provides utility.

A reputation function which does give feedback, with $\mathcal{R} \subseteq \mathds{Z}$ and $H = H^\prime + T$ if $|H| \geq 1$, is the following:
\[ r(A|H) = \begin{cases}
0 & \text{if } H = \emptyset\\
r(A|H^\prime) + 1 & \text{if } T = (A, B), r(B|H^\prime) \geq 0\\
r(A|H^\prime) - 1 & \text{if } T = (\cdot, A)\\
r(A|H^\prime) & \text{else}.\\ 
\end{cases} \]
That is, $A$ gains a unit of reputation for a transaction with $B$, provided that $B$ has a nonnegative reputation. Being on the consuming end of a transaction costs a unit of reputation.

Because no agent will be awarded reputation for working for a free-rider (as $B$ did in the previous example. When $M$ is prescribed by the above $r$, all agents can consume without contributing only once; showing that $L_M$ is bounded above by $|M|$.

\subsection{Sybil resistance}\index{Sybil attack}
Because all agents create and maintain their own chain and are allowed to join the market freely, creating a new chain (and as such, a new identity) is cheap. As such, we can't assume all agents act independently; a single agent might create other, \emph{sybil} identities who secretly collaborate. Their goal is to accumulate utility for some subgroup of the identities, after which the other identities are often abandoned. From this point onward, we adapt our model to reflect this possibility.

As such, we need our reputation function to be \emph{sybil resistant}. With our previous definition, $r$ doesn't protect against the sybil attack. An agent $A \in M$ could create an infinite number of identities $A_0, A_1, \ldots$ and create with each identity a transaction recording the transfer of value without actually performing any work.

We will now adapt our reputation function to be sybil resistant. Previously agents had some freedom in choosing who to perform work for, allowing for the possibility of working for one's sybil identities. We will now remove this opportunity by awarding a unit of reputation only when performing work for the agent in $M$ with the highest reputation. Again, receiving work costs a unit of reputation. Assume again $\mathcal{R} \subseteq \mathds{Z}$ and $H = H^\prime + T$ if $|H| \geq 1$,
\[ r(A|H) = \begin{cases}
0 & \text{if } H = \emptyset\\
1 & \text{if } H = T = (A, \cdot)\\
r(A|H^\prime) + 1 & \text{if } T = (A, B), B = \argmax_X r(X|H^\prime)\\
r(A|H^\prime) - 1 & \text{if } T = (\cdot, A)\\
r(A|H^\prime) & \text{else}.\\ 
\end{cases} \]
Let's start our analysis of this reputation function by considering the first transaction $T_1 = (A, B)$ happening in $M$. After this transaction, we have $r(A|T_1) = 1$ and $r(B|T_1) = -1$. Under our adapted model, $B$ has the possibility to abandon his previous identity and chain. $B$'s alternative is to start anew with an empty chain and identity $B^\prime$ without history, and as such with reputation $r(B^\prime|T_1) = 0$. Clearly, $B$ prefers reputation $0$ over reputation $-1$ and will create such a new identity $B^\prime$. Because at this point, $B$ will have become inactive, we now have $B \not \in M$. Moreover, $\sigma_A = 1$ and as such $L_M(H) = 1$.

Because work performed will only be awarded reputation if it is performed for the agent with the highest reputation, this will be the only work performed. In this way, every transaction will remove a unit of reputation from agent with the highest reputation, repeatedly changing which agent has the highest reputation $r = 1$, all others having $r = 0$. We conclude that for this definition of $r$ we have $\lim_{|H| \to \infty} L_M(H) = 1$.

\section{Deciding on fairness}
Possibly, there are many possible reputation functions which achieve finite leakage. Because the chosen reputation function prescribes which transactions will take place and which history will form, the influence of this choice couldn't be greater. As such, it is important to consider what values one would like to capture in this function. Which behaviors should be rewarded? The reputation function discussed previously strictly requires contribution before consumption, for example.

Other reputation functions could possibly give those with a certain reputation more leverage or market power than others, for example. Other possibilities include advantages for those with long histories (early adopters), or advantages for those with an ability to invest outside resources. There is also the possibility that some reputation functions will result in isolated communities doing transactions among each other. One can also wonder whether this is desirable.